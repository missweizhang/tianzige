\documentclass[12pt, letterpaper]{article}

\usepackage[fontsize=26, textcolor=lightgray, gridcolor=green, pycolor=brown, mizige]{tianzige}
\usepackage[left=0.5in, right=0.5in, top=0.5in, bottom=0.7in]{geometry}
\usepackage[shortlabels]{enumitem}

\setCJKmainfont{KaiTi}
\newcommand{\nogrid}[1]{\grid[color=black,grid=0]{#1}}

\newenvironment{question}
{\color{blue} \fontsize{30}{30} \textbf}{}

\begin{document}
\fontsize{24}{24}
%\setlength{\parindent}{4em}
\setlength{\parskip}{1em}

\begin{question}
2. 数笔画,填空 (Count the strokes and fill in the blanks.)
\end{question}
\begin{enumerate}[(1),leftmargin=2\parindent]
\item “在”一共有 \grid{} 画,第二画是 \grid{}。
\item “教”一共有 \grid{} 画,第六画是 \grid{}。
\item “写”一共有 \grid{} 画,第四画是 \grid{}。
\item “画”一共有 \grid{} 画,第七画是 \grid{}。
\end{enumerate}


\begin{question}
3. 照例子写汉字 (Combine two parts to make characters after the model.)
\end{question}
\\ \\
 \phantom{x} \qquad 例: 木 + 交 = \grid{校} \qquad 氵+ 又 = \grid{} \qquad 讠 + 吾 =  \grid{} \\ \\
 \phantom{x} \qquad \phantom{例:} 宀 + 子 =  \grid{} \qquad 讠+ 卖 =  \grid{} \qquad 哥 + 欠 = \grid{} \\ \\

\begin{question}
4. 比一比,再组词语 (Compare and form phrases.)
\end{question}
\\
\begin{table}[h!]
\centering
\begin{tabular}{c c c c c}
\nogrid{语} \grid{}\grid{} & \nogrid{学} \grid{}\grid{}  & \nogrid{汉} \grid{}\grid{} 
& \nogrid{写} \grid{}\grid{}  & \nogrid{在} \grid{}\grid{} \\
\nogrid{说} \grid{}\grid{} & \nogrid{字} \grid{}\grid{}  & \nogrid{欢} \grid{}\grid{} 
& \nogrid{雪} \grid{}\grid{}  & \nogrid{石} \grid{}\grid{} \\
& & & \nogrid{学} \grid{}\grid{} & \nogrid{右} \grid{}\grid{} \\
\end{tabular}
\end{table}
\makeatother % make @ normal again

\begin{question}
5. 选词语填空 (Choose the right words to fill in the blanks.)
\end{question}
\\ \\
\phantom{x} \qquad 教 \qquad 写 \qquad 在 \qquad 读 \qquad 画
\begin{enumerate}[(1),leftmargin=2\parindent]
\item 老师 \grid{} 学校 \grid{} 我们 \grid{} 汉字。
\item 妈妈 \grid{} 家 \grid{} 我 \grid{} 儿歌。
\item 爷爷 \grid{} 花园 \grid{} 我 \grid{} 画儿。
\item 爸爸 \grid{} 家 \grid{} 方方 \grid{} 汉语。
\end{enumerate}

\begin{question}
2. 照例子组词语 (Form phrases after the model.)
\end{question}
\\ \\
\phantom{x} \qquad 例: \grid[color=black]{汉} 字 \qquad \grid{} 语 \qquad \grid{} 文 \\ \\
\phantom{x} \qquad \phantom{例:}  \grid{} 欢 \qquad 学 \grid{} \qquad 写 \grid{} \\ \\

\begin{question}
3. 照例子连一连,写汉字 (Link and write after the model.)
\end{question}
\\
\begin{table}[h!]
\centering
\begin{tabular}{p{0.1\textwidth} p{0.1\textwidth} p{0.1\textwidth} p{0.1\textwidth} p{0.1\textwidth}}
\nogrid{孝} & \nogrid{讠} & \nogrid{哥} & \nogrid{讠} & \nogrid{氵} \\
\\
\\
\nogrid{又} & \nogrid{吾} & \nogrid{卖} & \nogrid{欠} & \nogrid{攵} \\
\\
\grid{} & \grid{} & \grid{读} & \grid{} & \grid{} \\
\end{tabular}
\end{table}

\end{document}
